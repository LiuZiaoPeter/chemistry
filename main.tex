\documentclass[a4paper,oneside]{ctexbook}

\usepackage{titlesec}

\usepackage{anyfontsize}

\usepackage{chemfig}
\usepackage{mhchem}
\usepackage{mol2chemfig}

\title{个人化学笔记}
\author{( . .)}
\date{\today}

\begin{document}

	\frontmatter
	\maketitle
	\tableofcontents

	\mainmatter

	\chapter{高一}

	\section{化合物 / 材料}

	\ce{(Al(OH)_4)^-} 分子结构为
	\begin{center}\chemfig{HO-[:60,,2,2]_{\mcfminus}Al(-[:330,,2,1]OH)(-[:150,,2,2]HO)-[:60,,2,1]OH}\end{center}
	事实上,能形成两性氢氧化物的金属都能形成类似的四羟基合物,例如 \ce{(Zn(OH)_4)^{2-}}
	\begin{center}\chemfig{HO-[:60,,2,2]_{2\mcfminus}Zn(-[:330,,2,1]OH)(-[:150,,2,2]HO)-[:60,,2,1]OH}\end{center}
	还有 \ce{(Ga(OH)_4)^-},\ce{(In(OH)_4)^-},\ce{(Sn(OH)_6)^{2-}},\ce{(Pb(OH)_4)^{2-}},\ce{(Be(OH)_4)^{2-}},\ce{(Cr(OH)_4)^-}。

	\ce{CaH_2} 是一种白色晶体,是强力还原剂和储氢材料,与水反应 \ce{CaH_2 + 2H_2O -> Ca(OH)_2 + 2H_2 ^},可用作干燥剂。

	\ce{CaC_2}
	\begin{center}\chemfig{\mcfright{Ca}{^{2\mcfplus}}-[,3,,,draw=none]\mcfright{C}{^{\mcfminus}}~[:180,,,2]^{\mcfminus}C}\end{center}
	也叫\textit{电石},和水反应 \ce{CaC_2 + 2H_2O -> Ca(OH)_2 + 2C_2H_2 ^} 生成乙炔。

	\ce{Na_2CO_3} 是白色粉末,\ce{NaHCO_3} 是白色晶体。

	\ce{NO} 与 \ce{NO_2} 都是奇数个电子的化合物,因此化学性质活泼:\ce{2NO + O_2 -> 2NO_2},以及 \ce{3NO_2 + H_2O -> NO + 2HNO_3}。\ce{NO_2} 还会二聚成 \ce{N_2O_4}:
	\begin{center}\chemfig{^{\mcfminus}O-[:60,,2,2]_{\mcfplus}N(=[:120,,2]O)-[,,2]\mcfright{N}{_{\mcfplus}}(-[:300]\mcfright{O}{^{\mcfminus}})=[:60]O}\end{center}

	\textit{葡萄糖} \ce{C_6H_{12}O_6} 和\textit{葡萄糖酸根} \ce{(C_6H_{11}O_7)^-}
	\begin{center}\chemfig{O=[:30]-[:330](-[:270,,,1]OH)-[:30](-[:90,,,1]OH)-[:330](-[:270,,,1]OH)-[:30](-[:90,,,1]OH)-[:330]-[:30,,,1]OH}\end{center}
	\begin{center}\chemfig{O=[:90](-[:150,,,2]^{\mcfminus}O)-[:30](-[:90,,,1]OH)-[:330](-[:270,,,1]OH)-[:30](-[:90,,,1]OH)-[:330](-[:270,,,1]OH)-[:30]-[:330,,,1]OH}\end{center}
	都具备还原性。葡萄糖的还原性来自于醛基 \ce{-COH},著名的银镜实验 \ce{RCOH + 2(Ag(NH_3)_2)^+ + 3OH^- -> RCOO^- + 2Ag v + 2H_2O} 中,\ce{-COH} 中的 \ce{C} 从 \(+1\) 变成了 \(+3\) 价。葡萄糖酸根中的醇羟基 \ce{-OH} 也有弱还原性,例如 \ce{CH_3CH_2OH + CuO -> CH_3CHO + Cu + H_2O}。

	电磁屏蔽性优良的 \ce{Mg-Al} 合金可以防止手机电子元件的内部干扰。

	\ce{Al-Fe} 合金的熔点并不高。

	\ce{CCl_4}
	\begin{center}\chemfig{C(-[:90]Cl)(-[:210]Cl)(<[:300]Cl)(<:[:350]Cl)}\end{center}
	具有正四面体结构,是一种非极性分子,密度约为 \(1.59\ \mathrm{g/cm}^3\)。由相似相溶原理,他可以溶解 \ce{I_2},\ce{Br_2} 等非极性物质。

	\ce{Br_2} 在有机溶剂中显紫红色,在水中显橙黄色。\ce{Br^-} 在水中无色,不溶于有机溶剂。

	\textit{苯}
	\begin{center}\chemfig{=^[:180]-[:240]=^[:300]-=^[:60](-[:120])}\end{center}
	是一种有机溶剂,密度约为 \(0.88\ \mathrm{g/cm}^3\)。

	\ce{HF} 不易溶。

	\textit{同素异形体}指的是 \ce{O_2} 和 \ce{O_3} 的关系;\textit{同分异构体}指的是\textit{丁烷}和\textit{2-甲基丙烷}
	\begin{center}\chemfig{-[:30]-[:330]-[:30]}\quad\chemfig{-[:90](-[:150])-[:30]}\end{center}
	的关系;没有专有名词表示 \ce{H_2} 和 \ce{D_2} 的关系。

	\ce{Si} 和 \ce{SiO_2} 可以和碱反应:\ce{Si + 2OH^- + H_2O -> SiO_3^{2-} + H_2 ^},\ce{SiO_2 + 2OH^- -> SiO_3^{2-} + H_2O}。\ce{SiO_2} 是酸性氧化物,但是不和水反应生成酸,但是有 \ce{SiO_2 + 4HF -> SiF_4 ^ + 2H_2O} 可用于刻蚀玻璃。
	
	\textit{偏硅酸} \ce{H_2SiO_3} 酸性弱于 \ce{H_2CO_3},受热易分解 \ce{H_2SiO_3 ->T[\Delta] SiO_2 + H_2O},可以强酸制弱酸 \ce{SiO_3^{2-} + H_2CO_3 -> H_2SiO_3 v + CO_3^{2-}}。

	\section{反应}

	活泼金属的氧化物遇水生成碱,如 \ce{Na_2O + H_2O -> 2NaOH};而不活泼的金属则反之,如 \ce{Cu(OH)_2 ->T[\Delta] CuO + H_2O}。

	\ce{Na_2O_2 + CO_2} 实际上是三步反应,且需要水参与:
	\begin{center}
		\ce{Na_2O_2 + H_2O -> 2NaOH + H_2O_2}\\
		\ce{2H_2O_2 -> 2H_2O + O_2 ^}\\
		\ce{2NaOH + CO_2 -> Na_2CO_3 + H_2O}
	\end{center}

	漂白粉变质:\ce{Ca(ClO)_2 + H_2O + CO_2 -> CaCO_3 + 2HClO}。

	\ce{MnO_2 + 4HCl -> MnCl_2 + 2H_2O + Cl_2 ^} 展现了 \ce{HCl} 的酸性,因为 \ce{H^+} 有和 \ce{O^{2-}} 结合生成 \ce{H_2O}。

	\ce{C + H_2O ->T[高温] CO + H_2},水过量时有副反应 \ce{CO + H_2O ->T[高温] CO_2 + H_2}。

	卤素互化物,金属性强的显 \(+1\) 价,故金属性强的写前面。和碱反应时,金属性强的生成含氧酸根,金属性弱的生成卤素离子。例如 \ce{ICl + 2OH- -> IO- + Cl- + H2O}。

	碱金属在 \ce{O_2} 中燃烧分别主要生成 \ce{Li_2O},\ce{Na_2O_2},\ce{KO_2},\ce{RbO_2},\ce{CsO_2}。

	\ce{3Fe^{2+} + 2(Fe(CN)_6)^{3-} -> Fe_3(Fe(CN)_6)_2 v},颜色为\textit{藤氏蓝}(Turnbull's blue)。

	测定臭氧含量:利用 \ce{2KI + O_3 + H_2O -> I_2 + 2KOH + O_2} 将臭氧转化为 \ce{I_2},然后滴定\textit{硫代硫酸钠},利用 \ce{I_2 + 2S_2O_3^{2-} -> S_4O_6^{2-} + 2I^-} 即可计算 \ce{I_2} 的量。\ce{S_4O_6^{2-}} 叫做\textit{连四硫酸根}:
	\begin{center}\chemfig{^{\mcfminus}O-[:60,,2]S(=[:60]O)(=[:150]O)-[:330]S-[:30]S-[:330]S(=[:240]O)(=[:60]O)-[:330]\mcfright{O}{^{\mcfminus}}}\end{center}

	\section{实验}

	在气体出口接上单球干燥管,球体里面放上需要的干燥剂(如 \ce{NaOH + CaO}),可以防止外面的气体进入装置干扰实验。

	吸收 \ce{HCl(g)} 需要饱和的 \ce{NaHCO_3} 溶液,以确保长期的中和能力。

	分析实验需要考虑各种挥发性气体(\ce{H_2O},\ce{HCl})的影响。

	在试管中加热生成气体,会有气体残留在试管中。

	\ce{Ca^{2+}} 可用于鉴别 \ce{CO_3^{2-}} 和 \ce{HCO_3^-},因为 \ce{Ca(HCO_3)_2} 是可溶的。此外,无水 \ce{CaCl_2} 会和水结晶成 \ce{CaCl_2.2H_2O / CaCl_2.6H_2O},可用作强力干燥剂,但是也会吸收 \ce{NH_3} 形成 \ce{CaCl_2.8NH_3}。
	
	容量瓶的可用大小只有 \(1,2,5,10,25,50,100,250,500,1000,2000,5000\mathrm{mL}\)。因此制备 \(400\mathrm{mL}\ 1\mathrm{mol/L}\) 的溶液需要 \(0.5\mathrm{mol}\) 的溶质。

\end{document}
