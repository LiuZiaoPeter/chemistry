\documentclass[a4paper,oneside]{ctexbook}

\usepackage{titlesec}

\usepackage{anyfontsize}

\usepackage{chemfig}
\usepackage{mhchem}
\usepackage{mol2chemfig}

\title{个人化学笔记}
\author{( . .)}
\date{\today}

\begin{document}

	\frontmatter
	\maketitle
	\tableofcontents

	\mainmatter

	\chapter{高一}

	\ce{(Al(OH)_4)^-} 分子结构为 \chemfig{HO-[:60,,2,2]_{2\mcfplus}Al(-[:330,,2,1]OH)(-[:150,,2,2]HO)-[:60,,2,1]OH}。事实上,能形成两性氢氧化物的金属都能形成类似的四羟基合物,例如 \ce{(Zn(OH)_4)^{2-}}(\chemfig{HO-[0]Zn^{2-}(-[2]OH)(-[6]OH)-[0]OH})

	\chemfig{H_3C-[:30]N**6(-(=O)-(**5(-N(-CH_3)--N-))--N(-CH_3)-(=O)-)}

\end{document}
