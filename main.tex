\documentclass[a4paper,oneside]{ctexbook}

\usepackage{titlesec}

\usepackage{anyfontsize}

\usepackage{chemfig}
\usepackage{mhchem}
\usepackage{mol2chemfig}

\title{个人化学笔记}
\author{( . .)}
\date{\today}

\begin{document}

	\frontmatter
	\maketitle
	\tableofcontents

	\mainmatter

	\chapter{高一}

	\section{化合物}

	\ce{(Al(OH)_4)^-} 分子结构为
	\begin{center}\chemfig{HO-[:60,,2,2]_{\mcfminus}Al(-[:330,,2,1]OH)(-[:150,,2,2]HO)-[:60,,2,1]OH}\end{center}
	事实上,能形成两性氢氧化物的金属都能形成类似的四羟基合物,例如 \ce{(Zn(OH)_4)^{2-}}
	\begin{center}\chemfig{HO-[:60,,2,2]_{2\mcfminus}Zn(-[:330,,2,1]OH)(-[:150,,2,2]HO)-[:60,,2,1]OH}\end{center}
	还有 \ce{(Ga(OH)_4)^-},\ce{(In(OH)_4)^-},\ce{(Sn(OH)_6)^{2-}},\ce{(Pb(OH)_4)^{2-}},\ce{(Be(OH)_4)^{2-}},\ce{(Cr(OH)_4)^-}。

	
	
	\section{反应}

	\section{实验}

\end{document}
